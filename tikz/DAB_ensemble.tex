\documentclass{article}

\usepackage[latin1]{inputenc}
\usepackage{tikz}
\usetikzlibrary{calc}
\usepackage{tkz-euclide}
\tikzset{line/.style={draw, thick, -latex'}}
%\usetikzlibrary{shapes,arrows}
\begin{document}
\pagestyle{empty}


\begin{figure}
\begin{center}
\begin{tikzpicture}[node distance = 2cm, auto]
% Define block styles
\tikzstyle{decision} = [diamond, draw, fill=cyan!30, 
    text width=4.5em, text badly centered, node distance=3cm, inner sep=0pt]
\tikzstyle{start} = [rectangle, draw, fill=cyan!30, 
    text width=5em, text centered, rounded corners, minimum height=4em]
\tikzstyle{output} = [trapezium, draw, fill=cyan!30, 
    text width=4em, text centered, minimum height=4em, trapezium right angle=110, trapezium left angle=70]
\tikzstyle{block} = [rectangle, draw, fill=cyan!30, 
    text width=5em, text centered, minimum height=4em]
\tikzstyle{line} = [draw, -latex']
\tikzstyle{cloud} = [draw, ellipse,fill=red!20, node distance=3cm,
    minimum height=2em]
    % Place nodes
    \node [start] (init) {Start};
    \node [decision, below of=init] (null) {NULL erwartet?};
    \node [block, below of=null, node distance=3cm] (corr) {gleitende Korrelation};
    \node [decision, below of=corr, node distance=3cm] (peak) {Peak?};
    \node [block, below of=peak, node distance=2.5cm] (energy) {Energie- messung};
    \node [decision, below of=energy] (first) {Symbol $z_{1,k}$?};
    \node [output, below of=first, node distance=3cm] (tag) {Start des Frames in $T_G$};
    \node [block, right of=null, node distance=4.5cm] (skip) {überspringe $T_S$};
    \node [block, below of=skip, node distance=3cm] (corr2) {einmalige Korrelation};
    \node [decision, below of=corr2] (peak?2) {Peak?};
    \node [above of=null, node distance=2cm] (backtonull){};
    \node [output, below of=peak?2, node distance=3cm] (lost track){nicht mehr in Sync};
    
    \path [line] (init) -- (null);
    \path [line] (null) -- node {ja} (corr);
    \path [line] (null) -- node {nein} (skip);
    \path [line] (corr) -- (peak);
    \path [line] (peak) -- node {ja} (energy);
    \path [line] (energy) -- (first);
    \path [line] (first) -- node {ja} (tag);
    \path [line] (peak) -- node {nein} ++(2.5cm,0);
    \path [line] (first) -- node {nein} ++(2.5cm,0) |- (corr);
    \path [line] (skip) -- (corr2);
    \path [line] (corr2) -- (peak?2);
    \path [line] (peak?2.south) -- node {nein} (lost track);
    \path [line] (peak?2.east) -- node{ja}++(1cm,0) -- ++(0,8cm) -- ++(-6.5,0);
    \path [line] (tag) -- node{}++(0,-1cm) -- node {} ++(-2.5cm,0) -- node{}++(0,17.5cm) -- ++(2.5cm,0);
    \path [line] (lost track.south) -- ++(0,-1cm) -- ++(-2,0);
\end{tikzpicture}
\end{center}
\caption{Programmablaufplan der Zeitsynchronisation}
\label{chart:zeitsync}
\end{figure}
\end{document}