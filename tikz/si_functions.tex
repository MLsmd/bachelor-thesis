\documentclass{article}

\usepackage[latin1]{inputenc}
\usepackage{tikz}
\usetikzlibrary{calc}
\usepackage{tkz-euclide}
\tikzset{line/.style={draw, thick, -latex'}}
\usepackage{tikz}
\usepackage{mathtools}
\usepackage{amsmath}
\usepackage{amsfonts}
\usepackage{siunitx}
\usepackage{pgfplots}
\usepackage{filecontents}
\usepackage{tikz}
\usetikzlibrary{datavisualization}
\usetikzlibrary{positioning}
\usetikzlibrary{arrows,calc,fit}
\tikzset{line/.style={draw, thick, -latex'}}
\usepackage{ trfsigns}
\usepackage{tkz-euclide}
\usepackage{svg}
%\usetikzlibrary{shapes,arrows}
\begin{document}
\pagestyle{empty}
\newcommand{\iterations}{200}

\begin{tikzpicture}[node distance = 1cm, auto]
\begin{axis}[
domain=-5:5,
enlarge x limits=false,
samples = \iterations,
height=\textwidth*0.5,
width=\textwidth,
xticklabels={,,,-3$\Delta f$,-2$\Delta f$,$-\Delta f$,$0$,$+\Delta f$,$+2\Delta f$,$+3\Delta f$},
xlabel={Frequenz},
ylabel={$\frac{sin(x)}{x}$},
]
    \addplot [mark=none, thick, red] {sin(deg(x*pi))/(x*pi)};
    \addplot [mark=none, densely dotted] {sin(deg((x-1)*pi))/((x-1)*pi)};
    \addplot [mark=none, densely dotted] {sin(deg(((x+1)*pi)))/(((x+1)*pi))};
    \addplot [mark=none, densely dotted] {sin(deg((x-2)*pi))/((x-2)*pi)};
    \addplot [mark=none, densely dotted] {sin(deg(((x+2)*pi)))/(((x+2)*pi))};
    \addplot [mark=none, densely dotted] {sin(deg((x-3)*pi))/((x-3)*pi)};
    \addplot [mark=none, densely dotted] {sin(deg(((x+3)*pi)))/(((x+3)*pi))};
    \addplot [mark=none, very thick] {sin(deg(x*pi))/(x*pi) + sin(deg((x-1)*pi))/((x-1)*pi) + sin(deg(((x+1)*pi)))/(((x+1)*pi)) + sin(deg((x-2)*pi))/((x-2)*pi) + sin(deg(((x+2)*pi)))/(((x+2)*pi)) + sin(deg((x-3)*pi))/((x-3)*pi) + sin(deg(((x+3)*pi)))/(((x+3)*pi))};

\addplot+[only marks,mark=o,mark options={scale=2},text mark as node=true, dashed] coordinates {
	(0,0)    
    (1,0) 
    (2,0) 
    (3,0) 
    (-1,0) 
    (-2,0) 
    (-3,0)
};
\addplot+[only marks,mark=o,mark options={scale=2},text mark as node=true, dashed] coordinates {
	(0,1)
    (1,1) 
    (2,1) 
    (3,1) 
    (-1,1) 
    (-2,1) 
    (-3,1)
};

\end{axis}

\end{tikzpicture}
\end{document}