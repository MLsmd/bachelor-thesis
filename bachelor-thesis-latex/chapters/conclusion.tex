\chapter{Fazit}\label{chapter:conclusion}
In dieser Arbeit wurde ein standardkonformer, funktionsfähiger DAB/DAB+ Sender und Empfänger implementiert. Das enstandene Transceiver System ist ein Software Defined Radio und kann mit verschiedenen Hardwarekonfigurationen genutzt werden.\\
Der DAB Sender bietet die Möglichkeit ein individuell anpassbares DAB/DAB+ Ensemble mit bis zu 7 Kanälen und entsprechenden Metainformationen zu erzeugen und abzustrahlen.\\
Der Empfänger ist in der Lage ein DAB/DAB+ Ensemble zu empfangen. Er kann sowohl Metadaten lesen und anzeigen, als auch Audiokanäle decodieren und wiedergeben.
Sowohl der Sender als auch der Empfänger wurden auf ihre Funktionalität verifiziert. Sie sind zudem beide in Echtzeit lauffähig.\\
Im Weiteren wurde die Applikation \textit{DABstep} realisiert, die auf Grundlage des implementierten Transceivers ein grafisches Benutzerfrontend darstellt. Die Applikation ist benutzerfreundlich gestaltet, indem sie Prozesse automatisiert und Metadaten sinnvoll aufbereitet. Durch eine übersichtliche und einfache Strukturierung der grafischen Oberfläche kann die Applikation auch von Laien bedient werden. Der Entwicklermodus erlaubt es, zusätzliche nachrichtentechnisch interessante Details anzeigen zu lassen.\\

In der Evaluation wurde das System unter verschiedenen Bedingungen auf seine Funktionalität und Performanz getestet. Ein Test mit Referenzdaten sowie ein Schleifentest verifiziert die grundsätzliche Funktionalität des Transceivers.\\
Im simulierten AWGN Kanal liegt die Bitfehlerrate des OFDM Systems von DAB nahe an der theoretischen Untergrenze. In realistischen Rayleigh-Fading Kanalmodellen mit verschiedenen Mehrwegempfangssituationen wurden deutlich höhere Bitfehlerraten gemessen. Die Paketfehlerraten des Gesamtsystems zeigen jedoch, dass eine Übertragung mit geringen Ausfallwahrscheinlichkeiten dank effektiver Kanalcodierung auch in diesen Kanälen bis zu Gewissen Grenzen möglich ist.\\
Für die Zukunft ist eine weitere Performanzsteigerung des Empfängers anzustreben, die durch eine zusätzliche Fehlerkorrektur im Firecode erreicht werden könnte. Zudem kann die Transceiver Applikation um weitere Funktionen, wie zum Beispiel das paketweise Übertragen von Bilder, erweitert werden.\\

Die Ergebnisse zeigen, dass das implementierte DAB/DAB+ System vergleichbare Performanz und Robustheit wie kommerzielle Geräte liefert. Ein großer Vorteil bietet die Realisierung als Software Definded Radio. Sie ermöglicht den Betrieb eines sehr konstengünstigen DAB/DAB+ Transceivers, zum Beispiel durch die Verwendung eines DVB-T Dongles. Der gesamte Transceiver kostet somit, abzüglich des Computers, unter 10\euro. Im Vergleich kosten enstprechende DAB+ Autoradios in Deutschland das 10 - 100 fache \cite{web:dab_im_auto}.\\

Im Vergleich zum UKW-Rundfunk bietet DAB/DAB+ eine flexiblere Gestaltung der Senderstruktur, mehr Möglichkeiten für Service Informationen und eine bessere, da rauschfreie, Audioqualität. Außerdem ist die DAB Übertragung deutlich energieeffizienter, weil die digitale Übertragung eine Audiokompression zulässt, die wiederum die Bandbreite pro Kanal reduziert.\\
Zusammenfassend ist der DAB/DAB+ Standard daher ein sinnvoller und notwendiger Nachfolger der UKW Übertragung. Demgegenüber stehen aktuelle Umfragen über die Nutzung und Verbreitung von DAB Radios, die weit hinter den erwarteten Zahlen zurückbleiben. So verfehlen aktuelle Nutzungszahlen von 35\% \cite{web:dab_in_uk} in Großbritannien Prognosen, die diese Abdeckung bereits für 2008 vorhergesagt hatten \cite{dab:ausblick}. Gründe für die schleppende Adaption von DAB ist neben den hohen Gerätepreisen und der bisher unzureichenden Empfangsabdeckung auch die Trägheit der Nutzer sich auf neue Übertragungsstandards einzustellen. Ein Gegenbeispiel bilden Norwegen und die Schweiz, die bereits dieses Jahr mit der Abschaltung der UKW-Rundfunkübertragung beginnen. Hier zeigt sich der Vorteil von DAB+, auch bei sehr schlechten Übertragungsbedingungen eine gute Audioqualität zu liefern, was in diesem Fall die Nuzter direkt betrifft und das System dadurch attraktiv macht.