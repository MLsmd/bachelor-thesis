\chapter{Implementierung eines DAB/DAB+ Transceivers}
\section{GNU Radio}
GNU Radio ist eine kostenlose und weitverbreitete Open-Source Software zur Programmierung und Ansteuerung von \ac{SDR}. Software Radio bezeichnet dabei ein Funksystem, welches die Signalverarbeitung anstatt durch Hardware (integrierte Schaltkreise) auf Softwareebene durchführt. Software hat den Vorteil, dass sie variabel und schnell austauschbar ist. Dadurch ist es möglich mit derselben Hardware viele verschiedene Funksysteme zu realisieren.\\
GNU Radio verwendet das Konzept von funktionalen Blöcken, die verbunden werden und damit eine Signalverarbeitungskette, einen sog. Flowgraph, bilden. Die einzelnen Blöcke führen jeweils eine spezifische Signalverarbeitungsoperation möglichst recheneffizient durch, weshalb ein GNU Radio Block in den meisten Fällen eine C++ Klasse darstellt. Die Ablaufsteuerung, also die Instanzierung und Anordnung der Blöcke in einem Gesamtsystem, geschieht in der Skriptsprache Python. Dazu werden die verwendeten C++ Klassen mit dem Programmierwerkzeug \ac{SWIG} nach Python übersetzt.\\
Neben der Möglichkeit, die GNU Radio Blöcke selbst zu entwerfen, gibt es schon eine umfangreiche Bibliothek an Signalverarbeitungsblöcken, die von grundlegenden Rechenoperationen bis hin zu komplexen Implementierungen wie ganzen Radarsystemen reichen. Die graphische Oberfläche \ac{GRC} ermöglicht es einen Flowgraph zu erstellen, ohne dabei Code schreiben zu müssen. Per Drag and Drop werden dazu die GNU Radio Blöcke angeordnet und mit Pfeilen verbunden. Der GRC ist ein nützliches Tool für schnelles Prototyping und für die Visualisierung eines Flowgraphs.\\

\begin{figure}[htb]
    \centering
    \includegraphics[width=0.8\textwidth]{figures/Gnuradio_logo.png}
    \caption{Logo von GNU Radio}
    \label{fig:gnu_radio_logo}
\end{figure}

Die in diesem Kapitel vorgestellte Realisierung eines DAB/DAB+ Transceivers wurde mit GNU Radio implementiert. Das dabei enstandene Out-Of-Tree Modul gr-dab enthält alle geschriebenen GNU Radio Blöcke und Flowgraphs und ist unter der \ac{GPL3} lizenziert und frei zugänglich auf der Plattform GitHub~\cite{repo:gr-dab} veröffentlicht. Die dabei verwendete Hardware ist unabhängig von der Implemtierung und kann von kostengünstigen DVB-T Dongles bishin zu gut verarbeiteten \acp{USRP} reichen.\\
In der Evaluation des Systems in Abschn.~\ref{sec:evaluation} wurden GNU Radio Flowgraphs zudem als Basis für Simulationen genutzt.

\section{Vorgehen bei der Implementierung}
Die Implementierung des DAB/DAB+ Transceivers in GNU Radio erfolgt nach dem Bottom-up Prinzip. Einzelne Algorithmen bzw. Funktionseinheiten werden jeweils in einem C++ Block implementiert. Jeder implementierte Block wird zuerst durch einen \ac{QA} Test auf seine korrekte Funktionalität überprüft, bevor er zum Einsatz kommen kann. Oftmals hat ein Block im Sender ein entsprechendes Gegenstück im Empfänger. Es ist sinnvoll dieses Blöcke zusammen zu implementieren und gegeneinander in einem Schleifentest (Loopback) zu testen. Aus diesem Grund werden der Sender und der Empfänger synchron implementiert. Aus den einzelnen Blöcken wird anschließend eine Signalverarbeitungskette aufgebaut. Um den GNU Radio Flowgraph übersichtlich zu halten, werden verschieden Gruppen von C++ Blöcken zu sogenannten hierarchischen Blöcken zusammengefasst. Ein hierarchischer Block ist ein GNU Radio Block in Python, der mehrere C++ Blöcke intern verbindet und somit zu einem Block zusammenfasst.\\
Um größere Funktionseinheiten zu testen, werden verifizierte Referenzdaten benötigt, die aus Mangel an Alternativen über eine Antenne von einer aktiven Radiostation in der Umgebung abgegriffen werden. Damit wird zuerst der DAB Empfänger gegen die aufgenommenen Referenzdaten getestet um anschließend den Sender gegen den nun verifizierten Empfänger zu testen.\\
Im Folgenden ist aus Gründen der Übersichtlichkeit die Implementierung des Senders und Empfängers getrennt erläutert.