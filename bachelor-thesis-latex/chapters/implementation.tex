\chapter{Implementierung eines DAB/DAB+ Transceivers}
\section{GNU Radio}
GNU Radio ist eine kostenlose und weitverbreitete Open-Source Software zur Programmierung und Ansteuerung von \ac{SDR}. Software Radio bezeichnet dabei ein Funksystem, welches die Signalverarbeitung anstatt durch Hardware (integrierte Schaltkreise) auf Softwareebene durchführt. Software hat den Vorteil, dass sie variabel und schnell austauschbar ist. Dadurch ist es möglich mit derselben Hardware viele verschiedene Funksysteme zu realisieren.\\
GNU Radio verwendet dabei das Konzept von funktionalen Blöcken, die verbunden werden und damit eine Signalverarbeitungskette, einen sog. Flowgraph, bilden. Die einzelnen Blöcke führen dabei jeweils eine spezifische Signalverarbeitungsoperation möglichst recheneffizient durch. Dabei stellt ein GNU Radio Block in den meisten Fällen eine eigene C++ Klasse dar. Die Ablaufsteuerung, also die Instanzierung und Anordnung der Blöcke in einem Gesamtsystem, geschieht in der Skriptsprache Python. Dazu werden die verwendeten C++ Klassen mit dem Tool \ac{SWIG} nach Python übersetzt.\\
Neben der Möglichkeit, die GNU Radio Blöcke selbst zu entwerfen, gibt es schon eine umfangreiche Bibliothek an Signalverarbeitungsblöcken, die von grundlegenden Rechenoperationen bis hin zu komplexen Implementierungen wie ganzen Radarsystemen reichen. Die graphische Oberfläche \ac{GRC} ermöglicht es einen Flowgraph zu erstellen, ohne dabei Code schreiben zu müssen. Per Drag and Drop werden dabei die GNU Radio Blöcke angeordnet und mit Pfeilen verbunden. Der GRC ist ein nützliches Tool für schnelles Prototyping und für die Visualisierung eines Flowgraphs.\\
Die in diesem Kapitel vorgestellte Realisierung eines DAB/DAB+ Transceivers wurde mit GNU Radio implementiert. Das dabei enstandene Out-Of-Tree Modul enthält alle geschriebenen GNU Radio Blöcke und Flowgraphs und ist unter \ac{GPL3} lizenziert und frei zugänglich. Die dabei verwendete Hardware ist unabhängig von der Implentierung und kann von kostengünstigen RTL-SDRs bishin zu gut verarbeiteten \ac{USRP} reichen.\\
In der Evaluation des Systems in Abschn.~\ref{sec:evaluation} wurden GNU Radio Flowgraphs als Basis für Simulationen genutzt.
\begin{figure}[htbp]
  \centering
  \includesvg{figures/Gnuradio_logo.svg}
  \caption{svg image}
\end{figure}
