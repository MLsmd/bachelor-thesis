\chapter{Einführung}
\label{sec:intro}
\ac{DAB} ist ein Übertragungsstandard zur Verbreitung von digitalem Radio. Das System wurde in den 90er Jahren im Zuge des Eureka 147 DAB Projekts entwickelt und kommt seit 1997 in vielen Teilen Europas und Asiens zum Einsatz. Mit dem DAB+ System, das eine Weiterentwicklung von DAB darstellt, soll langfristig die analoge \acp{UKW} Übertragung ersetzt werden.\\
Die digitale Datenübertragung bietet dabei eine verbesserte Audioqualität, da Übertragungsfehler durch Codierung empfängerseitig korrigiert werden können und somit nicht zu einem verrauschten Audiosignal wie bei FM führen. Eine Audiokompression verringert zusätzlich die Datenrate, was sich in einer gesteigerten Frequenzeffizienz pro Audiokanal widerspiegelt. Die digitale Übertragung bietet außerdem viele neue Möglichkeiten der medialen Unterstützung in Form von Service Informationen wie Albumcovers, ausführlichen Stauinformationen oder Wetterkarten.\\
DAB Sender sind in sog. Ensembles strukturiert. Ein DAB Ensemble enthält ein ganzes Multiplex an Audio- und Datenkanälen. Radiosender dieses Ensembles greifen auf eine Auswahl dieser Kanäle zu. Dieser dynamische Multiplex erlaubt eine flexible und individuelle Programmgestaltung. \\
DAB stellt vier verschiedene Übertragungsmodi zur Verfügung, die für verschiedene Ausbreitungsszenarien und Frequenzbereiche ausgelegt sind. Der Fokus liegt dabei auf dem mobilen Empfang bei terrestrischer Übertragung, es existiert aber auch jeweils ein Modus für die Satellitenübertragung, sowie die niederfrequente Übetragung per Kabel. Eine Besonderheit bei DAB stellt der Einsatz von sog. Single Frequency Networks (SFNs) dar, bei denen eine Vielzahl von örtlich getrennten Sendestationen auf der gleichen Frequenz ausstrahlen. Dadurch kann ein enormer Gewinn an Frequenzeffizienz erzielt werden. Der Einsatz von SFNs wird bei DAB durch eine hohe Robustheit des Übertragungssystems gegenüber Mehrwegeempfang ermöglicht \cite{dab_buch}.\\
Ziel dieser Arbeit ist die Implementierung und Evaluation eines Senders und Empfängers für DAB/DAB+. Die Implementierung wird dabei in Form eines Software Radios erfolgen, einem System bei dem ein Großteil der Signalverarbeitung auf Softwareebene durchgeführt wird. Dazu wird die Open-Source Software GNU Radio verwendet. Im Weiteren soll eine grafische Transceiver Applikation realisiert werden, die auf Basis des implementierten Senders und Empfängers eine benutzerfreundliche Oberfläche für das Senden und das Empfangen von DAB/DAB+ Signalen darstellt.

